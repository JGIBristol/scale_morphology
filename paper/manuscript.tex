\documentclass[11pt,a4paper,notitlepage]{article}

\usepackage[utf8]{inputenc}
\usepackage{amsmath}
\usepackage{graphicx}
\usepackage{mathtools}
\usepackage{hyperref}
\usepackage{natbib}

\author{Richard Lane, Carran Johal, Rabia Sevil, Chrissy Hammond}
\title{
	Scale Shape Methods
}

\begin{document}
\maketitle

% To make the building work while I don't have any references
\cite{example2023}

\section{Materials}
\begin{itemize}
	\item Stuff that happened in the lab
	\item Which fish we used
	\item How many we ended up with
	\item Imaging, staining etc.
	\item Flowchart of fishes
	\item I used the 878 scales in Carran/Postgrad/ALP_WT_flattened/
\end{itemize}

\section{Segmentation}
\begin{itemize}
	\item Show an image of a scale and segmentation mask
	\item SAM steps
		\begin{itemize}
			\item Prior
			\item ALP prior - rough segmentation, pick random points, find bbox
			\item Confocal prior - just pick some points in the middle of the image. Not used here.
			\item SAM model citation
		\end{itemize}
	\item These segmentations are in Carran/Postgrad/ALP_WT_flattened_segmentations/
	\item Manual cleaning
	\item Cleaned segmentations are in Carran/Postgrad/flattened_segmentations_cleaned/
\end{itemize}

\section{Elliptic Fourier Descriptors}
\begin{itemize}
	\item Maths - EFD used to describe the boundary of the shape
	\item Citation for code
	\item My workflow:
	\begin{itemize}
		\item Remove any spurious bits, fill holes - even the clean segmentations sometimes have issues
		\item Find equally spaced points around the edge of the segmentation
		\item Re-order points such that we have a consistent ordering.
			  Otherwise two rotated but otherwise identical shapes will have different EFD descriptors.
			  In my case, I reorder such that the first point is closest to the CoM of the object.
		\item Fit to these with pyefd - get coeffs
		\item Unscaled coeffs, since I care about the size and we want to extract raw features from the coeffs
		\item You can rotate these coeffs to remove things like rotation information (but the features I calculate later are rotation independent)
			  and it means the coefficients can't be easily used to plot the shape of the scale (which we might want to do,
			  to check that the EFD descriptors actually look like our scale)
	\end{itemize}
	\item What we get out
	\item A note on EFD - a complete description of shape.
	\begin{itemize}
		\item Normally people would just chuck things into PCA and hope for the best
		\item It will find the best linear* combinations that describe the dataset*
		\item This is fine
		\item There also seems to be a belief that PCA magically finds the right thing, and that you
		      might as well chuck your image pixels into PCA to find interesting features. This isn't true.
		\item We can come up with better (possibly non-linear, see the bumpiness metric below) features
			  that 1. actually describe the biologically interesting features in the dataset and
			  2. might never be picked up by pca (e.g. because they are non-linear, or because they don't describe
			  much global variation but do describe a lot of the difference between our classes).
		\item EFA is not a list of features derived from our shape - it is a *complete description of it* (down to some length scale)
		\item This means that ANY (reasonable) feature can be calculated from the EFA coefficients - it might be hard to work out how to do this, but it should be possible
		\item We show feature selection using EFDs here as an illustration - you could of course get e.g. size
			  by just counting pixels
	\end{itemize}
\end{itemize}

\section{Feature Extraction}
\begin{itemize}
	\item Features chosen:
	\begin{itemize}
		\item Size
		\item Aspect Ratio
		\item Bumpiness
		\begin{itemize}
			\item Why take the log? Because it makes it look right. Not taking the log squishes everything up to small values like 0.0001 vs 0.000001,
				  the absolute difference between which isn't that descriptive - but the log-difference is.
			\item Mathematical explanation: we take the log because we expect the Fourier power to decrease as (at least) some power law:
			\begin{itemize}
				\item the scale has finite area/perimiter/is bounded - this tells us that x is square-integrable, i.e. $\int_0^{2\pi}x(t)^2dt$.
				\item By Parseval's theorem, this means that $\sum_n^\infty(a_n^2 + b_n^2)$ (since the a, b coeffs came from the x points). Same for the $b, c$ coeffs from $y$.
				\item i.e. total harmonic power must converge - $\sum_{n=1}^\infty \left(a_n^2 + b_n^2 + c_n^2 + d_n^2 \right) < \infty$
				\item This tells us that the higher harmonics have to $\rightarrow 0$ as $n\rightarrow \infty$.
				\item How fast they decay is determined by smoothness - if the shape is continuous (but not differentiable, i.e. it has sharp corners) then
					  our coefficients will decay as $1/n$.
				\item Further constraints (like having smooth, non-sharp corners or being analytic) will mean the coeffs decay even faster.
				\item For scales, we can at least say that the Fourier power per harmonic decays at least like $1/n^2$ (but probably faster).
				\item The bumpiness (fraction of power in higher harmonics) is:
				\begin{equation}
					\frac{\sum_n^{nmax} a_n^2}{\sum_1^{nmax} a_n^2}
				\end{equation}
				\item Taking the log of this, we get
				\begin{equation}
					log\left(\sum_n^{nmax} a_n^2\right) - log\left(\sum_1^{nmax} a_n^2\right)
				\end{equation}
				\item Since we expect these to decay as power-laws, we have turned our power-law variation into linear
					  variation and turned multiplicative variation into additive variation (more intuitive)
				\item This is too much detail for the paper it's just interesting from an EFA perspective
			\end{itemize}
		\end{itemize}
	\end{itemize}
    \item[]
	\item Why not just use PCA - global variation instead of class separability, only linear, not biologically informed
\end{itemize}

\section{Statistical Tests + Interpreting Axes}
\begin{itemize}
	\item LDA explanation + sklearn citation
	\item LDA accuracy metrics
	\item Stats tests explanations - not too much detail
\end{itemize}

\section{Results}

\bibliographystyle{unsrt}
\bibliography{references}

\end{document}
