\documentclass[11pt,a4paper,notitlepage]{article}

\usepackage[utf8]{inputenc}
\usepackage{amsmath}
\usepackage{graphicx}
\usepackage{mathtools}
\usepackage{hyperref}
\usepackage{natbib}

\author{Richard Lane, Rabia Sevil, Chrissy Hammond}
\title{
    paper title
}

\begin{document}
\maketitle

% To make the building work while I don't have any references
\cite{example2023}

\section{Methods}
\subsection{Materials}
\begin{itemize}
	\item Stuff that happened in the lab
	\item Which fish we used
	\item How many we ended up with
	\item Flowchart of fishes
\end{itemize}

\subsection{Elliptic Fourier Analysis}
\subsubsection{Segmentation}
\begin{itemize}
	\item Pipeline, maybe with a flowchart
	\item Classical steps
	\item SAM steps
	\item Manual cleaning
\end{itemize}

\subsubsection{Elliptic Fourier Decomposition}
\begin{itemize}
	\item Maths
	\item Citation for code
	\item Maybe a diagram showing the pipeline
	\item How did I choose how many points to use - power spectrum? Haussdorf distance? area using shoelace formula?
    \item How did I choose how many harmonics to use - Hausdorff distance?
    \item Mention that EFA is useful because it regularises! There might be rough edges, EFA smoothes these sensibly...
\end{itemize}

\subsubsection{Dimensionality Reduction + Interpreting Axes}
\begin{itemize}
	\item PCA explanation
	\item LDA explanation
	\item sklearn citation
    \item[]
    \item Looking at the extreme ones
	\item Eigenshape analysis
	\item Direct axis interpretability - by plotting "pure" axes
\end{itemize}

\section{Results}

\bibliographystyle{unsrt}
\bibliography{references}

\end{document}
