% filepath: /home/mh19137/scale_morphology/paper/A_appendix_efa_area.tex

\documentclass[11pt,a4paper]{article}

\usepackage[utf8]{inputenc}
\usepackage{amsmath}
\usepackage{hyperref}

\author{Richard Lane}
\title{Appendix: Calculating the Area of a Curve Described by Elliptic Fourier Analysis}

\begin{document}

\maketitle

\label{app:efa-area}
Given a description of a closed curve in terms of its elliptic Fourier descriptors (EFDs), the
area of the enclosed curve can be calculated trivially from its coefficients.

Recall the definitions of our EFDs:
\begin{equation}
	\label{eqn:EFDs}
	\begin{aligned}
		x(t) & = a_0 + \sum_{n=1}^{N}
		\left(
		a_n\cos(nt) + b_n\sin(nt)
		\right)                       \\
		y(t) & = c_0 + \sum_{n=1}^{N}
		\left(
		c_n\cos(nt) + d_n\sin(nt)
		\right)
	\end{aligned}
\end{equation}
which allows us to encode a shape as an $\left(N\times4\right)$ list of coefficients $\left\{a_n,b_n,c_n,d_n\right\}$.

We will start from Green's theorem in the plane. For any vector field $\vec{v}$ and closed contour $C$:
\begin{equation}
	\label{eqn:green_theorem}
	\oint_C \vec{v} \cdot d\vec{l} = \iint_D (\nabla \times \vec{v}) \cdot d\vec{A}
\end{equation}
If we choose $\vec{v} = \frac{1}{2}\left(-y, x\right)$, we find that the curl on the RHS evaluates to 1.

This gives us the area $A$:
\begin{equation}
	\label{eqn:area_int}
	A = \frac{1}{2}\oint_C \left(-y, x\right) \cdot \left(dx, dy\right)
\end{equation}
However, for a curve $C$ described by our EFDs we know $x$ and $y$ from Eq.~\ref{eqn:EFDs},
and can write down expressions for the line elements $dx = \frac{dx}{dt}dt$ and $dy = \frac{dy}{dt}dt$
in terms of the parameter $t$:

\begin{equation}
	\label{eqn:EFDs_derivative}
	\begin{aligned}
		dx & = dt\sum_{n=1}^{N}
		n\left(
		-a_n\sin(nt) + b_n\cos(nt)
		\right)                 \\
		dy & = dy\sum_{n=1}^{N}
		n\left(
		-c_n\sin(nt) + d_n\cos(nt)
		\right)
	\end{aligned}
\end{equation}

Multipyling Eq.~\ref{eqn:area_int} out using the differentials in Eq.~\ref{eqn:EFDs_derivative} gives us a long
expression. However, many of the terms cancel since we are integrating over the complete domain of t from $0$ to $2\pi$.
Terms involving the constant terms $a_0$ and $c_0$ cancel out, since they are integrals over a complete period of the sines and cosines,
which have zero mean. Cross-terms involving sines and cosines of different frequencies also cancel out.
For the same-frequency terms, all terms like $\sin(nt)\cos(nt)$ cancel out pairwise (e.g. there are terms like
$\left[a_nc_n\cos(nt)\sin(nt) - a_nc_n\cos(nt)\sin(nt)\right]$), which leaves us with:

\begin{equation}
	\label{eqn:EFD_area_sums}
	\begin{aligned}
		A & = \frac{1}{2}\int_0^{2\pi}\sum_{n=1}^Nna_nd_n(\sin^2(nt) + \cos^2(nt)) - nb_nc_n(\sin^2(nt) + \cos^2(nt)) \\
		  & = \frac{1}{2}\int_0^{2\pi}\sum_{n=1}^Nna_nd_n - nb_nc_n \\
          & = \pi\sum_{n=1}^Nn(a_nd_n - b_nc_n)
	\end{aligned}
\end{equation}
This is a remarkable result - we can find the area of any shape described by EFDs just from a sum of the coefficients.
\bigbreak
Intuitively, this makes sense - it makes sense that we can get the area of a closed curve without having to any calculations about their interior,
because a closed curve ``remembers'' the path it has taken and should know what is inside it.
The reason the EFD result is neat is because we chose an orthogonal basis for our expansion - all the cross-terms
cancel when we calculate the area, even though they may have complicated interactions (which don't cancel!) when describing the outline of a complex shape.
It also makes sense that the $a$/$c$ and $b$/$d$ coefficients are paired up - the coefficients
$a$ and $b$ encode the ellipse in the $x$-direction and $c$ and $d$ for the $y$-direction. The quantity derived in
Eq.~\ref{eqn:EFD_area_sums} looks like the magnitude of a cross-product; geometrically, this magnitude tells us about
the area of a parallelogram defined by the vectors in the cross-product. It is a similar result to Parseval's Theorem, which
states that the energy of a signal is just the sum of the Fourier coefficients squared, but we have just moved to 2D and replaced
energy with area.
\bigbreak
This is useful as it means we can rapidly find the area of different shapes represented by truncated series of EFDs,
which we want to do to find how many harmonics are required to make a good approximatioin of a shape.

\end{document}